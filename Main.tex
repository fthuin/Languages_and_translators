\documentclass[12pt, a4paper]{article}

\usepackage[english]{babel}
\usepackage[utf8]{inputenc}
\usepackage[T1]{fontenc}
\usepackage{booktabs}
\usepackage{pifont} % for cmark or xmark
\usepackage{xcolor}
\newcommand{\cmark}{\textcolor{green!65!black}{\ding{51}}}
\newcommand{\xmark}{\textcolor{red!90!black}{\ding{55}}}

\author{Florian Thuin}
\title{LINGI2132 --- Languages and translators}

\begin{document}
    \maketitle
  \section{Introduction}
  \subsection{Introduction}

  This course will be divided in two parts: Compilers and Domain Specific
  Languages (DSL). \newline

  Compilers for all programming languages share the same principles, knowing
  them helps to understand how every programming language works. Often, new
  programming languages share the syntax, semantics and ideas and understanding
  the underlying principles will help you to learn new programming languages.
  \newline

  Famous computer scientists work(ed) on compilers:

  \begin{description}
    \item[Fortran]: John Backus* (1957)
    \item[Lisp]: John McCarthy \& Steve Russell (1958)
    \item[C]: Dennis M. Ritchie* (1972)
    \item[C++]: Bjarne Stroustrup (1979)
    \item[Smalltalk]: Alan Kay* (1980)
    \item[Python]: Guido Van Rossum (1991)
    \item[Java]: James Gosling (1995)
    \item[Ruby]: Yukihiro Matsumoto (1995)
    \item[Scala]: Martin Odersky (2003)
  \end{description}

{\tiny
\noindent\begin{tabular}{lp{0.06\linewidth}p{0.08\linewidth}p{0.07\linewidth}p{0.09\linewidth}p{0.05\linewidth}p{0.08\linewidth}p{0.08\linewidth}p{0.1\linewidth}p{0.07\linewidth}}
    Language    & Fortran     & Lisp          & C                 & C++               & Smalltalk & Python           & Java          & Ruby               & Scala \\
    \toprule
    Inventor    & John Backus & John McCarthy & Dennis M. Ritchie & Bjarne Stroustrup & Alan Kay  & Guido Van Rossum & James Gosling & Yukihiro Matsumoto & Martin Odersky \\
    \midrule
    Year        & 1957        & 1958          & 1972              & 1979              & 1980      & 1991             & 1995          & 1995               & 2003 \\
    \midrule
    Compiled    & \cmark{}    & \xmark{}      & \cmark{}          & \cmark{}          & \cmark{}  & \cmark{}         & \cmark{}      & \xmark{}           & \cmark{} \\
    \midrule
    Interpreted & \xmark{}    & \cmark{}      & \xmark{}          & \xmark{}          & \xmark{}  & \cmark{}         & \cmark{}      & \cmark{}           & \cmark{} \\
    \midrule
    Static      & \cmark{}    & \xmark{}      & \cmark{}          & \cmark{}          & \xmark{}  & \xmark{}         & \cmark{}      & \xmark{}           & \cmark{} \\
    \midrule
    Dynamic     & \xmark{}    & \cmark{}      & \xmark{}          & \xmark{}          & \cmark{}  & \cmark{}         & \xmark{}      & \cmark{}           & \xmark{} \\
    \midrule
    on a VM     &             &               & \xmark{}          & \xmark{}          & \cmark{}  & \cmark{}         & \cmark{}      & \cmark{}           & \cmark{} \\
    \bottomrule
\end{tabular}
}

    \subsection{Compiler}
  % TODO : Ajouter l'image du compilateur
    \subsubsection{Lexical analysis}
    \subsubsection{Parsing/syntax analysis}
    \subsubsection{Semantic analysis}
    \subsubsection{Optimization}
    \subsubsection{Code generation}
    \subsection{Interpreter}
  % TODO : Ajouter l'image de l'interpréteur

  \section{Lexical analysis}
  \paragraph{Exercise}
  How many tokens ?
  \begin{center}
      \textcolor{blue}{x}
      \textcolor{green!60!black}{=}
      \textcolor{red}{0}
      \textcolor{green!60!black}{;}
      \verb#\n\t#
      \textcolor{gray}{while}
      \textcolor{green!60!black}{(}
      \textcolor{blue}{x}
      \textcolor{green!60!black}{<}
      \textcolor{red}{10}
      \textcolor{green!60!black}{)}
      \textcolor{green!60!black}{\{}
      \verb#\n\t#
      \textcolor{blue}{x}
      \textcolor{green!60!black}{++}
      \textcolor{green!60!black}{;}
      \verb#\n#
      \textcolor{green!60!black}{\}}
  \end{center}
  W : whitespace,
  \textcolor{gray}{K : keyword},
  \textcolor{blue}{I : identifier},
  \textcolor{red}{N : number},
  \textcolor{green!60!black}{O : other tokens}.
  \section{Parsing}
  \section{Semantic analysis}
  \section{JVM and code generation}
  \section{Garbage collection}
  \section{Introduction to Scala}
  \section{By-name parameters}
  \section{Traits}

\end{document}
